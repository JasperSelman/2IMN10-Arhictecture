\section{Model 4 - Collaboration Diagram}
\begin{enumerate}
	\item The model has an actor, which is the customer, and several objects in the object-oriented programming language that the system is programmed in.
	All of these are conceptual.
	
	\item The connections seem to be function calls on the objects. These are conceptual.
	
	\item
	\begin{description}
		\item[Development view] The objects that are to be programmed, and the functions they should call are visible. This is useful for the programmer.
		\item[Process view] The model gives an overview of the information flow of the program via the function calls and the objects. 
		System integrators and testers can see how the parts interact, and again, programmers know which functions to call.
	\end{description}
	
	\item
	\begin{description}
		\item[Performance/Scalability] No.
		\item[Availability/Reliability] No.
		\item[Security] No.
		\item[Maintainability] Yes, the model gives a clear description of the components in the system, and how they interact. 
		Using the model, it is clear to see which components are affected when extending/updating the system.
	\end{description}
	
	\item No.
	
	\item The model is pretty clear, as it is rather basic.
	It does not seem to have been drawn by conventions, but the actions and the components are clear enough to convey the meaning.
	This is also due to clear function and component names.
	However, in the case of the development view, the model leaves some things to be desired.
	There is no mention of class variables or other class functions apart from usages by other objects.
\end{enumerate}