\section{Model 7 - Vending Machine}
\begin{enumerate}
	\item The states, sub-states and super-states.
	They are all conceptual and model in which state the vending machine is after an event has occurred.
	
	\item The arrows represented by the events.
	These are all conceptual.
	They model an event which would happen in real life.
	
	\item
	\begin{description}
		\item[Process View] System integrators, testers can see how the various parts interact.
		Developers see what the machine has to do when something happens. They also see what they have to implement to see how the state changes.
		\item[Development View] The state-chart is used to create the model by developers.
	\end{description}
	
	\item
	\begin{description}
		\item[Performance/Scalability] No.
		\item[Availability/Reliability] Yes, there are error methods when an error occurs.
		\item[Security] No.
		\item[Maintainability] Yes, the chart is very clear and easy to maintain.
		It is both useful for implementation and modification.
	\end{description}
	
	\item No.
	
	\item It is a very clear diagram.
	In development view it is clear in which states the machine can be and how these states can be reached.
	It is up to the programmer as to how this is implemented.
	It is also very clear whether something is a state or an action.
	These are the only two possibilities in the chart.
	Furthermore are the sub-states very useful to see which components communicate with each other.
\end{enumerate} 