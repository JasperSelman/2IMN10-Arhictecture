\section{Model 5 - LinkedIn}
\begin{enumerate}
	\item The blocks in this model are the applications, the services, the ``cloud'', and all the databases and servers.
	The databases and servers are all physical and of course for storage of all the LinkedIn data.
	The applications, services and cloud are all conceptual. The applications represent the profiles of different users.
	The services represent what is used on LinkedIn, reading the news, communicate with other users, sharing something in a group, etc.
	The cloud is where all these web apps are used.
	
	\item There is only one sort connector in this case, namely an arrow.
	Sometimes the arrow literally states what happens between the components (for example profile updates).
	Other times it is less obvious and you only know to components are connected and communicate.
	All the connectors are conceptual.
	
	\item 
    \begin{description}
        \item[Logical view] You can clearly see which components should interact with other components and if it should be a read/write action or something else. 
        This is useful to system engineers.
        \item[Physical view] You can clearly see all the components in the view and it gives an overview of the services.
        This is beneficial to system engineers.
        \item[Development view] There is some notion of implementation details, for example the http-rpc and jms calls.
        This is useful to programmers, as they get a general idea of how the components are supposed to interact.
    \end{description}
	
	\item
	\begin{description}
		\item[Performance/Scalability] Yes, you see a stash of databases, and that every service has its own database.
		It is shown that this is extendible by the ... etc.
		\item[Availability/Reliability] No.
		\item[Security] No.
		\item[Maintainability] Yes, since every component has its own database, you can easily change some of the services or add/remove one.
	\end{description}
	
	\item Yes, you can clearly see the distribution between services and databases and which application uses which service.
	
	\item You can clearly see all the components in the sites and how they are connected.
	Different components have there own colouring and smaller parts of the model.
	Communication is clearly shown with arrows, sometimes the arrows also note what the communication is.
	Not really clear where the architecture starts (I guess in the web apps).
	But over all very structured and easy readable and understandable.
\end{enumerate} 