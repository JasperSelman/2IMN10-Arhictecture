\section{Model 2 - VICSDA}
\begin{enumerate}
	\item The building blocks are system services, it is the logical model of a virtual community.
	Thus, all of these building blocks are conceptual.
	
	\item The connectors are relationships between these building blocks, denoting in how the services use one another.
	
	\item
	\begin{description}
		\item[Development view] The structure of the program/database is visualized, such that the programmer has a clear view of how the program should be organized, and in what way the different parts use/depend on one another.
		\item[Process view] It is to some extent visible how the different entities interact, via the text that is written next to the relations. This is useful for system integrators.
	\end{description}
	
	\item 
	\begin{description}
		\item[Performance/Scalability] No.
		\item[Availability/Reliability] No.
		\item[Security] No.
		\item[Maintainability] Yes, the overview of the structure of the program is given, so the influence of altering/adding parts can be analysed.
	\end{description}
	
	\item No.
	
	\item The model is very clear. It is using standard drawing conventions, and the lines can be clearly distinguished, even when crossing each other.
	You can easily see the relations between the different entities in the model.
	However, it does seem some information is missing. For the development view, it would be better to also include functions. It also seems like not all variables are declared.
\end{enumerate}
