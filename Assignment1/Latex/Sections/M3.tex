\section{Model 3 - AUTOSAR}
\begin{enumerate}
	\item The building blocks we see in this model are all the components in the RTE system.
	They are divided in several categories, namely pink blocks for drivers, green for abstractions, blue for services.
	We also see some grey blocks.
	These are external OS-applications or Micro-controllers.
	At last we also have blue blocks.
	These are ``actors'' in the error throw part.
	So they are Receivers of the errors or Senders.
	All these building blocks are conceptual.
	
	\item The connectors we see are the arrows from certain parts to other parts indicating the errors which are thrown by the hardware and/or software.
	They are all conceptual.
	
	\item
    \begin{description}
        \item[Development view] We see a bit of an development view. The is useful for programmers, as they can see the basic structure of the program, and which errors to throw.
	    All the parts which have to be implemented in the RTE are shown, but how they are implemented and what they do however is left for the developer.
        \item[Process view] We also see a bit of the process view, mostly how the errors have to be handled and from which part they came. This is somewhat useful to testers, as they can wether errors are thrown correctly.
    \end{description}
	
	\item
	\begin{description}
		\item[Performance/Scalability] Yes, they try to address the performance.
		Libraries are used and every part of the software, like communication, memory, I/O and on-board devices have their own abstraction, drivers and services.
		I do not know however how well this model scales when you enlarge it.
		\item[Availability/Reliability] Yes, a lot of errors are handled and pointed out.
		\item[Security] No.
		\item[Maintainability] No.
	\end{description}
	
	\item A little bit.
	We know that parts of the software, like memory, I/O, communication, etc. have their own services, abstractions and drivers, but we do not really know how this is distributed in the RTE so we do not know a lot.
	
	\item I think this is not a very clear diagram, because it looks kind of messy at the start.
	This is because there are a lot of blocks and lines all over these blocks and on the right is some extra text.
	But when you look a little longer at the diagram you see that actually it is not that bad at all.
	All the blocks which have something in common, like OS, drivers, services etc. are coloured in the same way.
	Errors are always shown as a red lightning flash and the text on the right is the legend for the errors.
	Besides that I am missing the context of the model.
	I know that it has something to do with an automotive system from the URL and that a lot of errors are handled (or at least mentioned) but I do not really know what the system does.
\end{enumerate} 