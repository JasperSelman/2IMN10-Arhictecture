\section{Model 8 - Consumer Website}
\begin{enumerate}
	\item The model is divided into 4 sections, each of which contains several nodes that seem to model access to and the contents of a certain website, including backend. There is a Java servlet, a database, etc.
	All of these are conceptual.
	
	\item The connections show both the flow of the program as the protocol used. They are all conceptual.
	
	\item
	\begin{description}
		\item[Development view] Different subsystems are listed, along with access protocols. This is beneficial to programmers, to get an overview of the system to be implemented.
		\item[Process view] There is a notion of information flow in the connectors, along with the execution environment, as it lists the protocols used.
		This may come in handy for programmers, as they can look up how to program the protocols at hand.
	\end{description}
	
	\item
	\begin{description}
		\item[Performance/Scalability] No.
		\item[Availability/Reliability] No.
		\item[Security] No.
		\item[Maintainability] Yes, the main systems, subsystems, and interactions between these are all given.
	\end{description}
	
	\item Yes, there is a notion of distribution, as there are multiple tiers in the system.
	
	\item The model is moderately clear. 
	Instead of adhering to standard drawing conventions, the model defines its own shapes and lists them below.
	It is divided into different tiers, which adds to the clarity, but both the irregular shapes as the sheer number of different type components make the system unnecessarily complex.
	Additionally, there is a vast number of arrows, all differing in length, shape, and direction, which makes the model rather messy.
	On top of that, the arrows cross a lot of times.
	As such, it is difficult to understand the flow of the model.
\end{enumerate}