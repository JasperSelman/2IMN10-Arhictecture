\section{Model 6 - NHIN}
\begin{enumerate}
	\item The model shows several actors, like the user, provider, consumer, which are conceptual. 
	Then there are several records and systems, such as the gateway, the public health system, the warehouse, which are physical.
	There is a distinction between a local organisation's existing system-collection, and the NHIN system-collection, which denotes that there is a data flow between an organisation and perhaps a larger (national?) database, to which more organisations can connect.
	In the centre is a large node, saying ``connect'', the use of which is a little unclear. 
	It seems to be a central interface through which all of the systems in order to pass requests to other systems, which I assume to be conceptual.
	
	\item The arrows represent the interactions between systems and other systems, and systems and users.
	The arrow denotes the direction of data flow.
	
	\item
	\begin{description}
		\item[Logical view] The model denotes per user which actions/information are available to them.
		\item[Process view] The model shows the different components of the system, and the information flow between them. This is useful to system integrators.
	\end{description}
	
	\item
	\begin{description}
		\item[Performance/Scalability] No.
		\item[Availability/Reliability] No.
		\item[Security] Yes, the system makes notion of authorization and authentication.
		\item[Maintainability] No.
	\end{description}
	
	\item Yes, multiple systems are visible, a distinction is made between a local organisation and a public system. 
	There are multiple machines all exchanging data with each other.
	
	\item The model is moderately clear. At first hand, it is hard to notice what is going on, because of the multitude of arrows, all in close proximity.
	Additionally, where some of the components are very clearly labelled, it is unclear what the ``connect'' node in the centre entails.
	However, the actions do seem to make sense, and there is a clear view on which actors are present and how they can interact with the system.
\end{enumerate}