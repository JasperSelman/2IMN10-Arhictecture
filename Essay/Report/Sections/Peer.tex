\section*{Requirements of Peer-to-Peer based Massively Multiplayer Online Gaming}

The last part we were going to discuss about the architecture presented in this paper is the performance of it. Think of properties like , maintenance, load distribution, security, availability, usability and scalability. However the architecture presented in this paper, the peers@play project, is still in development and therefore it is hard to already reason on its performance, but we will still try to do it, but not in a way that we will take a look at every property, but more in a way that we take a look at what is already there in the project and how this might perform.\\
\indent The architecture proposed in \cite{peer} is in fact a communication engine which is structured in a communication middleware and a distribution framework. This communication middleware has two API's to distribute data. One for a proactive world state propagation and one for a reactive world state propagation. This is likely to increase the usability of the engine, since multiple types of games fit in this middleware. The distribution framework gives support for distributed game management and computation, the usability also benefits from this.\\
\indent The authors state in section \emph{Peer-to-Peer Networking} that they are currently bus implementing a way to construct a\emph{k-connected} graph. If they succeed in doing this, the availability, reliability and scalability would benefit greatly from it. With such a network the system is able to add people to a network or remove people from it. This represents people logging in and off. If this is possible the system is also able to ensure that the game state of players does not depend on how many other players are online and who goes offline while you are playing. For adding a person to a graph, the authors are currently busy how to add new people to a graph instead of creating a new graph. If and when this works the system is also very well scalable. People could just log in and they are connected to an existing graph.\\
\indent At the moment the authors are also busy investigating how the game world changes have to be distributed over all the peers. If they do this well enough than the load distribution of the game is quite well, but they have to be careful. If the load per peer gets to high, the performance for that single peer might go down a lot. They have also a reactive world state propagation, this propagation is used to do research in virtual crime investigation in the game (which shows us they are thinking about the security of their architecture). Unfortunately we cannot say more about this part since there are no real decisions made yet how to handle this game state changes.\\
\indent The last part the authors are currently busy with tis the consistency of the game. For the game view of the user it is important the game is consistent in such a way that is player A picks up an item, player B cannot pick up this item anymore. At the moment the authors developed a model for ordering of events in a system to be correct, but nothing of this is yet implemented. \\
\indent Overall the system has great potential regarding these performance properties, but it was still in progress so we had to wait for the result how well it is implemented in practice. Since the paper was quite old we tried to find how this peer@play project ended and if it was a success or not, but unfortunately we were not able to find any information on this. So this might insinuate that the architecture in practice was not so good.
