\section*{Towards Service Oriented Architecture for MMOG}
The final architecture we will discuss is the one described in the paper \emph{Towards Service Oriented Architecture for MMOG} \cite{service}.
The main focus of the paper is to propose a service oriented architecture for massive multiplayer online games, that ensures a certain quality of service and provides scalability.
Quality of service mostly seems to be expressed in terms of real-time interaction.
The authors have different measures for real-time interaction, ranging from `non real-time' to `extreme real-time'.
In user communication, the writes attempt to achieve `hard real-time' interaction.
However, some of the other requirements of the system seem to collide with the quality of service requirements. \\

As is apparent from the title, the architecture proposed in the article is aimed at games.
In this case, apart from only focussing on game state, the authors have considered other services in a game as well, for example, the account login service, or the account billing service.
These services are modelled by the service-oriented architecture (SOA) concept.
The SOA ensures functionality like scalability, re-usability, monitoring and tracking of information, and others. \\
Based on this concept, the architecture provides processes for user authentication, user checks, and credit card checks.
These processes are implemented using Business Process Execution Language (BPEL).
BPEL is very well suited to the processes above, while also having a very low deployment cost.
It is even capable of real time services.
However, this is where it clashes with the quality of service requirements. \\
The language can only provide `soft real-time' interaction, one step below the hard real-time interaction we need.
Hence, it does not have the capabilities for peer to peer interaction as far as games go. \\

To solve this, the proposed architecture employs a form of middleware to facilitate peer to peer interactions, separate from the BPEL implementation.
The article proposes the use of an open source `Data Distribution Service (DDS) Standard'.
This standard provides an information hub where users can dynamically connect in order to publish and subscribe to each other's data as needed, in an efficient, high-performance manner, with minimal overhead.
The flexibility of this communication exchange ensures the quality of service requirement.
Additionally, having users publish data to subscribed users themselves saves a lot in terms of needed communication hardware. 
As a result, the proposed system scores very well in the scalability field, since the producer only has to provide the middleware where users can connect to each other. \\

The architecture seems very well thought out in terms of scalability and quality of service.
Splitting the architecture in two parts with slightly different requirements ensures that the system as a whole is more efficient.
The split makes sense, as well.
The SOA concept ensures that the authentication, user check, and credit card check processes remain scalable, with the added benefit that BPEL has a very low deployment cost.
DDS also seems to be very suitable for having users send data to each other. \\

Using DDS says something about how the architecture handles the game state.
Users subscribe to other users for their data, and users publish their data to every user subscribed to them.
However, the article does not state how users are to determine to which users they should subscribe, or what/when they should publish data.
Supposedly, this should be handled by the game engine.
In that manner, each user has it's own portion of the game state, only being subscribed to the users that have data important to that particular user.
This does give rise to some questions regarding security. 
Users may be able to influence the data they send, or just not send data at all. 
All of these might result in inconsistencies in the intersection of game states between users, which might put some users at a disadvantage.\\
However, inconsistencies in the game states do not even have to be deliberate.
Users with a bad internet connection may also affect the game state of other users. 
As per the example given in the article, in a race game, multiple users may think they're at the number one position, due to delayed messages from other players regarding their position on the track.
This may be addressed by game developers, but it does seem like placing a burden upon them, which should actually have been solved in the architecture itself. \\

Now we are going to analyze the architecture proposed in \cite{service} regarding to performance properties like maintenance, load distribution, security, availability, usability and scalability. 
First we would like to discuss the load distribution.
The proposed architecture is split into two parts, where one part is responsible for less network intensive processes, such as user authentication and credit card checks. 
The other part is where the brunt of the communication is done.
With DDS as middleware, the users are able to subscribe to other users for relevant data, and publish data to their own subscribers.
Since DDS only serves as a means for users to subscribe to each other, the load is distributed over all of the users.
As a result, the load is very well distributed, even if the number of users increases, as is common in P2P architectures. 
Moreover, even the other part of the architecture is very open to improvements as needs may change.
The SOA concept calls for loosely coupled entities that are independant.
As such, the system can easily be extended with other services as a chat service.
Thus, even as needs may change in the future, load balancing can still be guaranteed. \\
Now let’s take a closer look at the usability. 
The decoupling discussed above together with the BPEL language proposed in section \emph{Simulation flow in BPEL} of \cite{service} seems to give a great boost for usability. 
The BPEL has great functionality when the processes involved do not have a hard real time constraint. 
It is also stated that this language is a standard for defining processes. 
So this insinuates that there the tool is very usable (usable in the fact that it is easy to use and can model a lot of different processes). 
When there are however a lot of processes involved which do require real time constraints, this solution does not work. 
In that case the authors state that they have to use the DDS standard. 
This standard is open source and dynamic. 
This implies that the system is usable as a middleware system. \\
In terms of availability, the SOA part of the system scores very well.
All of the services are independent of each other, so if one part of the system fails, the others can still function.
In the current implementation, this does not hold much ground, as user authentication is generally a vital part of massive multiplayer online games.
However, if the system was to be extended with a chat server, for example, the game would still be able to function, even if the chat server was unavailable.
Additionally, this helps in maintenance, as several services may be temporarily disabled while updating, without having to shut down the entire game. \\
The availability of the DDS part of the architecture is ambiguous. 
Any user can be disconnected without any problem, the game will still function for the other players.
However, if the DDS server allowing users to subscribe to each other would fail, the structure would crumble.
At first users will still be able to play, as they're already connected to other players.
With the server down, however, they will be unable to connect to new users, so eventually, the game states of each user will become inconsistent. \\

The security of the architecture is not discussed in detail by the article.
It is noted that the SOA concept allows for monitoring and tracking of information.
This could be used to prevent malicious information from entering the services, but in the actual implementation it is not discussed. \\
In the user communication, security is not discussed either.
It seems to be up to the game developers to prevent users from sending malicious data to each other, and to make sure the users connect to the appropriate users.
So overall it is hard to say how secure this architecture is, because this part is not really handled in the paper.\\
The last property we are going to discuss is the scalability.
Obviously, the DDS part of the architecture is highly scalable, as users are only dependant on each other for data, instead of a central server.
The SOA part of the system is also very much scalable. 
It allows for integration of new components, resources, and technologies for system expandability.
This is further confirmed by the fact that the services are loosely coupled, meaning that they are independent.