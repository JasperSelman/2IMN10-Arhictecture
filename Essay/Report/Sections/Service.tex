\section*{Towards Service Oriented Architecture for MMOG}
The final architecture we will discuss is the one described in the paper \emph{Towards Service Oriented Architecture for MMOG} \cite{service}.
The main focus of the paper is to propose a service oriented architecture for massive multiplayer online games, that ensure a certain quality of service and provide scalability.
Quality of service seems mostly seems to be expressed in terms of real-time interaction.
The authors have different measures for real-time interaction, ranging from `non real-time' to `extreme real-time'.
In user communication, the writes attempt to achieve `hard real-time' interaction.
However, some of the other requirements of the system seem to collide with the quality of service requirements. \\

As is apparent from the title, the architecture proposed in the article is aimed at games.
In this case, apart from only focussing on game state, the authors have considered other services in a game as well, for example, the account login service, or the account billing service.
These services are modeled by the service-oriented architecture (SOA) concept.
The SOA ensures functionality like scalability, re-usability, monitoring and tracking of information, and others. \\
Based on this concept, the architecture provides processes for user authentication, user checks, and credit card checks.
These processes are implemented using Business Process Execution Language (BPEL).
BPEL is very well suited to the processes above, while also having a very low deployment cost.
It is even capable of real time services.
However, this is where it clashes with the quality of service requirements. \\
The language can only provide `soft real-time' interaction, one step below the hard real-time interaction we need.
Hence, it does not have the capabilities for peer to peer interaction as far as games go. \\

To solve this, the proposed architecture employs a form of middleware to facilitate peer to peer interactions, separate from the BPEL implementation.
The article proposes the use of an open source `Data Distribution Service (DDS) Standard'.
This standard provides an information hub where users can dynamically connect in order to publish and subscribe to each other's data as needed, in an efficient, high-performance manner, with minimal overhead.
The flexibility of this communication exchange ensures the quality of service requirement.
Additionally, having users publish data to subscribed users themselves saves a lot in terms of needed communication hardware. 
As a result, the proposed system scores very well in the scalability field, since the producer only has to provide the middleware where users can connect to each other. \\

The architecture seems very well thought out in terms of scalability and quality of service.
Splitting the architecture in two parts with slightly different requirements ensures that the system as a whole is more efficient.
The split makes sense, as well.
The SOA concept ensures that the authentication, user check, and credit card check processes remain scalable, with the added benefit that BPEL has a very low deployment cost.
DDS also seems to be very suitable for having users send data to each other. \\

Using DDS says something about how the architecture handles the game state.
Users subscribe to other users for their data, and users publish their data to every user subscribed to them.
However, the article does not state how users are to determine to which users they should subscribe, or what/when they should publish data.
Supposedly, this should be handled by the game engine.
In that manner, each user has it's own portion of the game state, only being subscribed to the users that have data important to that particular user.
This does give rise to some questions regarding security. 
Users may be able to influence the data they send, or just not send data at all. 
All of these might result in inconsistencies in the intersection of game states between users, which might put some users at a disadvantage.\\
However, inconsistencies in the game states do not even have to be deliberate.
Users with a bad internet connection may also affect the game state of other users. 
As per the example given in the article, in a race game, multiple users may think they're at the number one position, due to delayed messages from other players regarding their position on the track.
This may be addressed by game developers, but it does seem like placing a burden upon them, which should actually have been solved in the architecture itself. \\

\noindent Now we are going to analyze the architecture proposed in \cite{service} with regarding to performance properties like maintenance, load distribution, security, availability, usability and scalability. 
First we would like to discuss the load distribution. 
The architecture proposed in the paper showed us a that the overall system consists of many different decoupled modules. 
There are servers for accounts to login, to handle the billing process (which is of course to great importance for the vendor), auxiliary servers for services like chatting and of course the game servers which run the real game. 
So you might argue that the load is well distributed, but the real work is done on the game server. 
In the previously discussed paper the game was split into client services like the graphics and server services like the game state. 
This is now done all on this single server. 
So this load is not well distributed. 
It is also questionable how well this works when there are multiple game servers since the authors state in the section \emph{simulation of SOA Concept for Servers} that all servers are implemented as standalone servers. 
This implies there is on real communication between servers. \\
\indent Now let’s take a closer look at the usability. 
The decoupling discussed above together with the BPEL language proposed in section \emph{Simulation flow in BPEL} of \cite{service} seem to give a great boost for usability. 
The BPEL has great functionality when the processes involved do not have a hard real time constraint. 
It is also stated that this language is a standard for defining processes. 
So this insinuates that there the tool is very usable (usable in the fact that it is easy to use and can model a lot of different processes). 
When there are however a lot of processes involved which do require real time constraints, this solution does not work. 
In that case the authors state that they have to use the DDS standard. 
This standard is open source and dynamic. 
This implies that the system is usable as a middleware system.\\
\indent When we take a quick look at the availability we see that it is easy to keep the system running when parts of the game go down (like the chat), because of the different types of servers, since they are all separated. 
But it is difficult keep the game running when the game server goes down. 
Since there is no communication between other game servers it is hard to redistribute the load of a game server when one goes down. 
On top of that nothing is stated about backups or something, so if a server does go down we think the users lose everything (since their last login). \\
\indent However since all the services are placed on separate severs it might be easier to do maintenance on these parts. 
Also since the BPEL language is widely supported it might be easy to do maintenance on parts which are created with this language. 
The real question in this part his how easy is it to do maintenance systems following the DDS standard, however nothing is really stated about this standard we do not really know how well this adheres to the maintenance property.\\
\indent The security provided by this architecture is, that important information, like account information and payment (credit card) information is stored on separate servers. 
The DDS server also does not use the TCP protocol, which is well known and quite safe, but use their own protocol for information exchange. 
For this protocol we know nothing about how we have to authenticate ourselves or what we have to do to get data from this protocol. 
In the rest of the paper nothing can be found on what security measures are taken to prevent malicious users on running their own code in the server or how the servers which store the information are protected. 
So overall it is hard to say how secure this architecture is, because this part is not really handled in the paper.\\
\indent The last property we are going to discuss is the scalability. 
The architecture described with the standalone servers does not really seem scalable. 
When the number of users grows, does the number of servers grow or does the server get bigger? 
It seems unlikely that still one game server can be used to maintain all the information. 
The BPEL language does scale well with protocols, it is widely used and as long as the protocol does not have a real time constraint it can be modeled within this language. 
The DDS standard needs the proposed architecture to be scalable and the authors promise that their architecture is scalable to multiple users, but nowhere any evidence can be found which supports this claim. 
So it is really questionable how well this architecture scales, because it seems to not scale very well.